% ---------------------
% Pacotes OBRIGATÓRIOS
% ---------------------
\documentclass[
    % -- opções da classe memoir --
    12pt,                    % tamanho da fonte
    twoside,                 % para impressão em verso e anverso. Oposto a 
    openany,
    a4paper,                 % tamanho do papel. 
    % -- opções da classe abntex2 --
    chapter=TITLE,          % títulos de capítulos convertidos em letras maiúsculas
    section=TITLE,          % títulos de seções convertidos em letras maiúsculas
    subsection=TITLE,       % títulos de subseções convertidos em letras maiúsculas
    subsubsection=TITLE,    % títulos de subsubseções convertidos em letras maiúsculas
    % -- opções do pacote babel --
    english,                % idioma adicional para hifenização
    french,                 % idioma adicional para hifenização
    %spanish,               % idioma adicional para hifenização
    greek,
    latim,
    brazil                  % o último idioma é o principal do documento
    ]{article}
\usepackage{babel}                  % hifenização em português
\usepackage{fontenc}            % Seleção de códigos de fonte.
\usepackage[utf8]{inputenc}         % Codificação do documento (conversão automática dos acentos)
\usepackage{float}                  % flutuação do texto/figura
\usepackage{graphicx}               % Inclusão de gráficos
\usepackage{color}                  % Controle das cores
% \usepackage{xcolor}                 % molduras e caixas em cores
\usepackage{lastpage}               % Usado pela Ficha catalográfica
% \usepackage{indentfirst}            % Indenta o primeiro parágrafo de cada seção.
\usepackage{hyperref}               % Adicionar links clicáveis às referências
\usepackage{mdframed}               % colocar moldura
\usepackage{multicol}               % múltiplas colunas
\usepackage{epsfig,subfig}          % Inclusão de figuras
\usepackage{microtype}              % Melhorias de justificação
\usepackage{lmodern}                % Usa a fonte Latin Modern
\usepackage{imakeidx}               % Para criação de indice
%------------------------------------------------------------------------
% Formatação ABNT
%------------------------------------------------------------------------
\usepackage{setspace}               % Para permitir espaçamento simples, 1 1/2 e duplo
\usepackage[a4paper,left=3cm,top=3cm,right=2cm,bottom=2cm]{geometry}
\setmainfont{Luxurious Roman}

% ---------------------
% Pacotes ADICIONAIS
% ---------------------
\usepackage{fancyhdr}                      
\usepackage{lipsum}                     % Geração de dummy text
\usepackage{amsmath,amssymb,mathrsfs}   % Comandos matemáticos avançados 
\usepackage{verbatim}                   % Para poder usar o ambiente "comment"
\usepackage{tabularx}                   % Para poder ter tabelas com colunas de largura auto-ajustável
\usepackage{afterpage}                  % Para executar um comando depois do fim da página corrente
\usepackage{url}                        % Para formatar URLs (endereços da Web)
\usepackage{acronym}                    % Listagem e referência das SIGLAS
\usepackage{textcase}                   % Texto em letra MAIÚSCULA
\usepackage[portuguese]{datetime2}
\usepackage{subfiles}
\usepackage[dvipsnames,usenames]{xcolor}
\usepackage{tocbibind}                  % Adiciona a seção de referências no sumário
\usepackage{newfloat}
\usepackage{academicons}
\usepackage{glossaries}

% ---------------------
% Pacotes de CITAÇÕES
% ---------------------
\usepackage[brazilian,hyperpageref]{backref}    % Paginas com as citações na bibl
\usepackage[alf]{abntex2cite}                % Citações padrão ABNT (alfa) e/ou (numéricas)
\usepackage{footmisc}
% ---------------------
% ---------------------
% Configurações
% ---------------------

\DTMlangsetup{showdayofmonth=false}          % Para ocultar o dia do mês.

\hypersetup{
    colorlinks=true,
    linkcolor=blue,
    citecolor=Cyan,
    urlcolor=blue
}

% O tamanho da identação do parágrafo é dado por:
\setlength{\parindent}{0cm}

% Controle do espaçamento entre um parágrafo e outro:
\setlength{\parskip}{0.2cm}  % tente também \onelineskip


\renewcommand{\thesection}{\arabic{section}}

% ---------------------
% IMPORT DE CONFIGURAÇÕES
% ---------------------

%  ======== VARIAVEIS ========


% VARIÁVEIS - TEXTUAIS
\newcommand{\TITULO}{Macroeconomia do Brasil}
\newcommand{\SUBTITULO}{o processo de desindustrialização brasileira até os dias atuais}
\newcommand{\PCUM}{Macroeconomia}
\newcommand{\KWone}{}
\newcommand{\PCDOIS}{Industrialização}
\newcommand{\KWtwo}{}
\newcommand{\PCTRES}{Assimetrias Econômicas}
\newcommand{\KWthree}{}
\newcommand{\PCQUATRO}{}
\newcommand{\KWfour}{}
\newcommand{\PCCINCO}{}
\newcommand{\KWfive}{}

% VARIÁVEIS - PESSOAS
\newcommand{\DISCENTE}{Alanna Khésley Lopes da costa}
\newcommand{\DISCENTEDOIS}{Camila Santos Estrela}
\newcommand{\DISCENTETREZ}{Emily Cabral dos Santos Pinto}
\newcommand{\DISCENTEQUATRO}{Lucas Cristovam de Barros}
\newcommand{\DISCENTECINCO}{Nelis Nelson Arruda da Cruz Júnior}
\newcommand{\DISCENTESEIS}{Yasmini do Nascimento Oliveira}
\newcommand{\AUTOR}{\DISCENTE}
\newcommand{\AUTORDOIS}{\DISCENTEDOIS}
\newcommand{\AUTORTREZ}{\DISCENTETREZ}
\newcommand{\AUTORQUATRO}{\DISCENTEQUATRO}
\newcommand{\AUTORCINCO}{\DISCENTECINCO}
\newcommand{\AUTORSEIS}{\DISCENTESEIS}
\newcommand{\COAUTOR}{}
\newcommand{\ORIENTADOR}{Felipe Macedo Zumba}
% \newcommand{\COORIENTADOR}{Coorientador}
\newcommand{\MONITOR}{Eduardo Murilo Pinto Taborda}
% \newcommand{\SUPERVISORI}{Supervisor 1}   % Caso seja relatório de estágio
% \newcommand{\SUPERVISORII}{Supervisor 2}  % Caso seja relatório de estágio

% VARIÁVEIS - INSTITUIÇÕES
\newcommand{\MEC}{Ministério da Educação}
\newcommand{\UNIVERSIDADE}{Universidade Federal do Rio Grande do Norte}
\newcommand{\ESCOLA}{Escola de Ciências e Tecnologia}
% \newcommand{\EMPRESANOME}{Empresa)}     % Caso seja relatório de estágio
% \newcommand{\EMPRESA}{\acs{Sigla da Empresa}}   % Caso seja relatório de estágio

% VARIÁVEIS - CURSOS e DISCIPLINAS
\newcommand{\MEUCURSO}{Bacharelado Interdisciplinar em Ciências e Tecnologia}
\newcommand{\ENFASE}{Negócios Tecnológicos} % Caso o curso seja de 2 ciclos
\newcommand{\DISCIPLINA}{Arranjos Produtivos e Tecnológicos}

% VARIÁVEIS - LUGARES
\newcommand{\LUGAR}{Natal/RN}

%  VARIAVEIS - TEMPO
\newcommand{\DATA}{{\thedate}}
\newcommand{\thedate}{{\today}}
\newcommand{\MTU}{\MakeTextUppercase}
% \newcommand{\PERIODODE}{{outubro de 2021 a julho de 2022}}  % Normalmente usado em relatórios de estágio



\makeindex  % Compila o índice
\makeglossaries

\begin{document}

% ----------------------------------------------------------
% ELEMENTOS PRÉ-EXTUAIS (Capa, Resumos, Abstract, Indice, Sumário, Silgas)
% ----------------------------------------------------------

% Define o estilo da página da capa
%  ======== VARIAVEIS ========


% VARIÁVEIS - TEXTUAIS
\newcommand{\TITULO}{Macroeconomia do Brasil}
\newcommand{\SUBTITULO}{o processo de desindustrialização brasileira até os dias atuais}
\newcommand{\PCUM}{Macroeconomia}
\newcommand{\KWone}{}
\newcommand{\PCDOIS}{Industrialização}
\newcommand{\KWtwo}{}
\newcommand{\PCTRES}{Assimetrias Econômicas}
\newcommand{\KWthree}{}
\newcommand{\PCQUATRO}{}
\newcommand{\KWfour}{}
\newcommand{\PCCINCO}{}
\newcommand{\KWfive}{}

% VARIÁVEIS - PESSOAS
\newcommand{\DISCENTE}{Alanna Khésley Lopes da costa}
\newcommand{\DISCENTEDOIS}{Camila Santos Estrela}
\newcommand{\DISCENTETREZ}{Emily Cabral dos Santos Pinto}
\newcommand{\DISCENTEQUATRO}{Lucas Cristovam de Barros}
\newcommand{\DISCENTECINCO}{Nelis Nelson Arruda da Cruz Júnior}
\newcommand{\DISCENTESEIS}{Yasmini do Nascimento Oliveira}
\newcommand{\AUTOR}{\DISCENTE}
\newcommand{\AUTORDOIS}{\DISCENTEDOIS}
\newcommand{\AUTORTREZ}{\DISCENTETREZ}
\newcommand{\AUTORQUATRO}{\DISCENTEQUATRO}
\newcommand{\AUTORCINCO}{\DISCENTECINCO}
\newcommand{\AUTORSEIS}{\DISCENTESEIS}
\newcommand{\COAUTOR}{}
\newcommand{\ORIENTADOR}{Felipe Macedo Zumba}
% \newcommand{\COORIENTADOR}{Coorientador}
\newcommand{\MONITOR}{Eduardo Murilo Pinto Taborda}
% \newcommand{\SUPERVISORI}{Supervisor 1}   % Caso seja relatório de estágio
% \newcommand{\SUPERVISORII}{Supervisor 2}  % Caso seja relatório de estágio

% VARIÁVEIS - INSTITUIÇÕES
\newcommand{\MEC}{Ministério da Educação}
\newcommand{\UNIVERSIDADE}{Universidade Federal do Rio Grande do Norte}
\newcommand{\ESCOLA}{Escola de Ciências e Tecnologia}
% \newcommand{\EMPRESANOME}{Empresa)}     % Caso seja relatório de estágio
% \newcommand{\EMPRESA}{\acs{Sigla da Empresa}}   % Caso seja relatório de estágio

% VARIÁVEIS - CURSOS e DISCIPLINAS
\newcommand{\MEUCURSO}{Bacharelado Interdisciplinar em Ciências e Tecnologia}
\newcommand{\ENFASE}{Negócios Tecnológicos} % Caso o curso seja de 2 ciclos
\newcommand{\DISCIPLINA}{Arranjos Produtivos e Tecnológicos}

% VARIÁVEIS - LUGARES
\newcommand{\LUGAR}{Natal/RN}

%  VARIAVEIS - TEMPO
\newcommand{\DATA}{{\thedate}}
\newcommand{\thedate}{{\today}}
\newcommand{\MTU}{\MakeTextUppercase}
% \newcommand{\PERIODODE}{{outubro de 2021 a julho de 2022}}  % Normalmente usado em relatórios de estágio



\newenvironment{capa}{
    \clearpage
    \thispagestyle{empty}
    \newgeometry{margin=0pt}
}{
    \restoregeometry
    \clearpage
}

% Início da capa
\begin{capa}
    \noindent % Remove indentação à esquerda
    \includegraphics[width=\paperwidth,height=\paperheight,trim=0 0 0 0,clip]{imagens/capa.png} % Imagem de fundo
\end{capa}
% Fim da capa

%  ======== VARIAVEIS ========


% VARIÁVEIS - TEXTUAIS
\newcommand{\TITULO}{Macroeconomia do Brasil}
\newcommand{\SUBTITULO}{o processo de desindustrialização brasileira até os dias atuais}
\newcommand{\PCUM}{Macroeconomia}
\newcommand{\KWone}{}
\newcommand{\PCDOIS}{Industrialização}
\newcommand{\KWtwo}{}
\newcommand{\PCTRES}{Assimetrias Econômicas}
\newcommand{\KWthree}{}
\newcommand{\PCQUATRO}{}
\newcommand{\KWfour}{}
\newcommand{\PCCINCO}{}
\newcommand{\KWfive}{}

% VARIÁVEIS - PESSOAS
\newcommand{\DISCENTE}{Alanna Khésley Lopes da costa}
\newcommand{\DISCENTEDOIS}{Camila Santos Estrela}
\newcommand{\DISCENTETREZ}{Emily Cabral dos Santos Pinto}
\newcommand{\DISCENTEQUATRO}{Lucas Cristovam de Barros}
\newcommand{\DISCENTECINCO}{Nelis Nelson Arruda da Cruz Júnior}
\newcommand{\DISCENTESEIS}{Yasmini do Nascimento Oliveira}
\newcommand{\AUTOR}{\DISCENTE}
\newcommand{\AUTORDOIS}{\DISCENTEDOIS}
\newcommand{\AUTORTREZ}{\DISCENTETREZ}
\newcommand{\AUTORQUATRO}{\DISCENTEQUATRO}
\newcommand{\AUTORCINCO}{\DISCENTECINCO}
\newcommand{\AUTORSEIS}{\DISCENTESEIS}
\newcommand{\COAUTOR}{}
\newcommand{\ORIENTADOR}{Felipe Macedo Zumba}
% \newcommand{\COORIENTADOR}{Coorientador}
\newcommand{\MONITOR}{Eduardo Murilo Pinto Taborda}
% \newcommand{\SUPERVISORI}{Supervisor 1}   % Caso seja relatório de estágio
% \newcommand{\SUPERVISORII}{Supervisor 2}  % Caso seja relatório de estágio

% VARIÁVEIS - INSTITUIÇÕES
\newcommand{\MEC}{Ministério da Educação}
\newcommand{\UNIVERSIDADE}{Universidade Federal do Rio Grande do Norte}
\newcommand{\ESCOLA}{Escola de Ciências e Tecnologia}
% \newcommand{\EMPRESANOME}{Empresa)}     % Caso seja relatório de estágio
% \newcommand{\EMPRESA}{\acs{Sigla da Empresa}}   % Caso seja relatório de estágio

% VARIÁVEIS - CURSOS e DISCIPLINAS
\newcommand{\MEUCURSO}{Bacharelado Interdisciplinar em Ciências e Tecnologia}
\newcommand{\ENFASE}{Negócios Tecnológicos} % Caso o curso seja de 2 ciclos
\newcommand{\DISCIPLINA}{Arranjos Produtivos e Tecnológicos}

% VARIÁVEIS - LUGARES
\newcommand{\LUGAR}{Natal/RN}

%  VARIAVEIS - TEMPO
\newcommand{\DATA}{{\thedate}}
\newcommand{\thedate}{{\today}}
\newcommand{\MTU}{\MakeTextUppercase}
% \newcommand{\PERIODODE}{{outubro de 2021 a julho de 2022}}  % Normalmente usado em relatórios de estágio


\newenvironment{reumo}{%
  % Código para iniciar o ambiente
}{%
  % Código para encerrar o ambiente
}

\section*{Resumo}
    % Escreva o resumo logo abaixo


    % Fim do resumo
\vspace{2cm}

\noindent\textbf{Palavras-chave}: \PCUM, \PCDOIS, \PCTRES, \PCQUATRO.

% %  ======== VARIAVEIS ========


% VARIÁVEIS - TEXTUAIS
\newcommand{\TITULO}{Macroeconomia do Brasil}
\newcommand{\SUBTITULO}{o processo de desindustrialização brasileira até os dias atuais}
\newcommand{\PCUM}{Macroeconomia}
\newcommand{\KWone}{}
\newcommand{\PCDOIS}{Industrialização}
\newcommand{\KWtwo}{}
\newcommand{\PCTRES}{Assimetrias Econômicas}
\newcommand{\KWthree}{}
\newcommand{\PCQUATRO}{}
\newcommand{\KWfour}{}
\newcommand{\PCCINCO}{}
\newcommand{\KWfive}{}

% VARIÁVEIS - PESSOAS
\newcommand{\DISCENTE}{Alanna Khésley Lopes da costa}
\newcommand{\DISCENTEDOIS}{Camila Santos Estrela}
\newcommand{\DISCENTETREZ}{Emily Cabral dos Santos Pinto}
\newcommand{\DISCENTEQUATRO}{Lucas Cristovam de Barros}
\newcommand{\DISCENTECINCO}{Nelis Nelson Arruda da Cruz Júnior}
\newcommand{\DISCENTESEIS}{Yasmini do Nascimento Oliveira}
\newcommand{\AUTOR}{\DISCENTE}
\newcommand{\AUTORDOIS}{\DISCENTEDOIS}
\newcommand{\AUTORTREZ}{\DISCENTETREZ}
\newcommand{\AUTORQUATRO}{\DISCENTEQUATRO}
\newcommand{\AUTORCINCO}{\DISCENTECINCO}
\newcommand{\AUTORSEIS}{\DISCENTESEIS}
\newcommand{\COAUTOR}{}
\newcommand{\ORIENTADOR}{Felipe Macedo Zumba}
% \newcommand{\COORIENTADOR}{Coorientador}
\newcommand{\MONITOR}{Eduardo Murilo Pinto Taborda}
% \newcommand{\SUPERVISORI}{Supervisor 1}   % Caso seja relatório de estágio
% \newcommand{\SUPERVISORII}{Supervisor 2}  % Caso seja relatório de estágio

% VARIÁVEIS - INSTITUIÇÕES
\newcommand{\MEC}{Ministério da Educação}
\newcommand{\UNIVERSIDADE}{Universidade Federal do Rio Grande do Norte}
\newcommand{\ESCOLA}{Escola de Ciências e Tecnologia}
% \newcommand{\EMPRESANOME}{Empresa)}     % Caso seja relatório de estágio
% \newcommand{\EMPRESA}{\acs{Sigla da Empresa}}   % Caso seja relatório de estágio

% VARIÁVEIS - CURSOS e DISCIPLINAS
\newcommand{\MEUCURSO}{Bacharelado Interdisciplinar em Ciências e Tecnologia}
\newcommand{\ENFASE}{Negócios Tecnológicos} % Caso o curso seja de 2 ciclos
\newcommand{\DISCIPLINA}{Arranjos Produtivos e Tecnológicos}

% VARIÁVEIS - LUGARES
\newcommand{\LUGAR}{Natal/RN}

%  VARIAVEIS - TEMPO
\newcommand{\DATA}{{\thedate}}
\newcommand{\thedate}{{\today}}
\newcommand{\MTU}{\MakeTextUppercase}
% \newcommand{\PERIODODE}{{outubro de 2021 a julho de 2022}}  % Normalmente usado em relatórios de estágio



\section*{Abstract}
    % Escreva a versão ingles o resumo (abstract) logo abaixo

    The article reviews the literature on the innovation ecosystem in Brazil and Industry 4.0 technologies. These technologies integrate advanced systems that automate, communicate and decide on production processes. In addition, the importance of innovation for the country's industrialization is highlighted, which faces challenges of resources and investments in STI. The literature review uses specific criteria to select materials published between 2018 and 2023 that support the discussion and innovative strategies for the innovation ecosystem in Brazil.
 
    % Fim do resumo   
\vspace{2cm}

\noindent\textbf{Keywords}: \KWone, \KWtwo,  \KWthree, \KWfour.

\newpage


\tableofcontents
\clearpage
% \newenvironment{Lista de Siglas}{%
  % Código para iniciar o ambiente
}{%
  % Código para encerrar o ambiente
}

\section*{Lista de Siglas e abreviaturas}
\addcontentsline{toc}{section}{Lista de Siglas e abreviaturas}
    \begin{acronym}[AAAAAAAAA]
    \setlength{\itemsep}{-\parsep}
    \acro{BCT}{Bacharelado em Ciências e Tecnologia}
    \acro{ECT}{Escola de Ciências e Tecnologia}
    \acro{MEC}{Ministério da Educação}
    \acro{UFRN}{Universidade Federal do Rio Grande do Norte}
    \acro{CTI}{Ciência, Tecnologia e Inovação}
    \acro{IoT}{Internet of Things}
    \acro{IA}{Inteligencia Artificial}
    \acro{CPS}{Cyber-Physical System}
    \acro{FHC}{Fernando Henrique Cardoso}
    \end{acronym}
\newpage

% ----------------------------------------------------------
% ELEMENTOS EXTUAIS (Introdução, Desenvolvimento, Conclusão)
% ----------------------------------------------------------

% \renewcommand{\abstractname}{} % Remove o título "Resumo"

\newenvironment{Introducao}{%
  % Código para iniciar o ambiente
}{%
  % Código para encerrar o ambiente
}

\begin{Introducao}
\section{MITOS E LENDAS RPG}

Mitos e Lendas é um sistema de RPG base, criado como uma base para ser jogado com várias mitologias pelo mundo. Nascido do compromisso de transformar o role play game, em uma experiência mais inclusiva e otimizada para pessoas neurodivergentes e/ou com necessidades especificas.

O sistema nasceu após uma brincadeira entre amigos, com a intenção de balancear o sistema até então utilizado em nossas mesas. A brincadeira virou verdade e Mitos e Lendas nasceu, trazendo inovação, diversão e inclusão para nossas mesas.

Se você recebeu essa versão do RPG, significa que vocês estava lá, quando a ideia surgiu. De algum modo você foi um dos motores necessários para o start do projeto e por isso. Gratidão! Esperamos que de coração, sua experiência com esse sistema seja absolutamente divertida e despreocupada. Sem mais delongas, bem vindos a mitos e lendas RPG.

Grande impacto que essa mitologia teve e tem, no imaginário de crianças, jovens e adultos pelo mundo. É um dos pilares da cultura ocidental, oferecendo referências para elaboração de séries, filmes, jogos e livros.

Iniciaremos nossas aventuras, explorando as possibilidades do panteão grego, ou quase isso. Os deuses os quais escolhemos explorar são: Afrodite, Apolo, Ares, Ártemis, Atena, Hades, Hécate, Hefesto, Hermes, Morfeu, Poseidon e Zeus.

A simbologia, a história, personalidade, relações, alianças e inimizades desses deuses, serão exploradas por meio das habilidades passadas a seus filhos e/ou devotos. Jogue como um filho de Atena, julgue ser digno de caçar com Ártemis, ou explore se entrege a escuridão do Tártaro como devoto de Hades. São muitas possibilidades que você jogador poderá vivenciar nas mãos do seu Mestre.
    
\end{Introducao}




\newenvironment{desenvolvimento}{%
  % Código para iniciar o ambiente
}{%
  % Código para encerrar o ambiente
}

\begin{desenvolvimento}
\section{Desenvolvimento}

\end{desenvolvimento}

\newenvironment{conclusao}{%
  % Código para iniciar o ambiente
}{%
  % Código para encerrar o ambiente
}

\begin{conclusao}
\section{Conclusão}


\end{conclusao}

% ----------------------------------------------------------
% ELEMENTOS PÓS-TEXTUAIS (Referências, Glossário, Apêndices)
% ----------------------------------------------------------


\include{caps/postextual/4-anexos}
\newpage

\newenvironment{glossario}{%
  % Código para iniciar o ambiente
}{%
  % Código para encerrar o ambiente
}

\begin{glossario}

    \newglossaryentry{cloudComputing}
    {
        name = {Computação em Núvem},
        description = {\\ é um termo coloquial para a disponibilidade sob demanda de recursos do sistema de computador, especialmente armazenamento de dados e capacidade de computação, sem o gerenciamento ativo direto do utilizado}
    }
    
    \newglossaryentry{govFHC}
    {
        name = {Governo \acro{FHC}},
        description = {\\ O Governo Fernando Henrique, também chamado Governo FHC, teve início com a posse da presidência por Fernando Henrique Cardoso, em 1° de janeiro de 1995, e terminado em 1° de janeiro de 2003, quando Luiz Inácio Lula da Silva assumiu a presidência.}
    }

\end{glossario}

\bibliography{referencias.bib}
% \clearpage


\printindex

\end{document}
